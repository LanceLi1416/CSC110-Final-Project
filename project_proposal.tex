\documentclass[fontsize=11pt]{article}
\usepackage{amsmath}
\usepackage[utf8]{inputenc}
\usepackage[margin=0.75in]{geometry}

\title{CSC110 Project Proposal: TODO FILL IN YOUR PROJECT TITLE HERE}
\author{Fardin Faruk, Sharon Hsieh, Sinan Li, Jeffery Zhan}
\date{Friday, November 5, 2021}

\begin{document}
    \maketitle

    \section*{Problem Description and Research Question}

    Severe acute respiratory syndrome coronavirus 2 (SARS-CoV-2), commonly known for causing the illness of COVID-19, emerged wildly after it was first discovered in late 2019 (Britannica, T. Editors of Encyclopaedia, 2021). Following its discovery, society as we knew it was upended and countries went into lockdown, millions suffered psychologically due to isolation, the stock market crashed, and more. Beyond that, countless regulations and restrictions have been placed upon us such as limited gatherings and social isolation, all of which have begun to test the patience of the general public. While COVID-19 has proven difficult to combat, Canada has put forth their best efforts to prevent the spread of the virus and thanks to this hard work, the country and its population has become one of the top vaccinated in the world (Mathieu et al., 2021). Despite this achievement, however, social restrictions do not appear to be leaving anytime soon, which begs the question: how long will society be stuck like this? A broad examination of a graph charting the daily cases of COVID-19 demonstrates high fluctuation in cases over certain intervals (Government of Canada, 2021).

    Our group aims to research and answer the question: \textbf{When will Canada have less than 270 COVID-19 cases in a day?} We believe once Canada hits this number, Canadians will find their lives fully free of masks and social restrictions. With this research, we will examine certain variables such as the number of vaccinated people, the rates of COVID-19 testing, and attempt to ascertain their connections with the COVID-19 cases. We hope our results will offer insight and possibly answers to the aforementioned question, including details on what factors affect COVID-19’s spread and how, and finally identify the day that Canada will reach less than 270 cases in a day. We chose 270 cases in particular because, according to the Centers for Disease Control and Prevention (CDC), the low incidence threshold is defined as 0.71 cases per 100,000 people every day (Spiro \& Gee 2020), and extrapolating that to the entire Canadian population of 38.1 million (Statistics Canada, 2021), we can take the number of 270.

    \section*{Dataset Description}

    TODO

    \section*{Computational Plan}

    TODO

    \section*{References}

% NOTE: LaTeX does have a built-in way of generating references automatically,
% but it's a bit tricky to use so we STRONGLY recommend writing your references
% manually, using a standard academic format like APA or MLA.
% (E.g., https://owl.purdue.edu/owl/research_and_citation/apa_style/apa_formatting_and_style_guide/general_format.html)

    \hangindent=0.7cm
    Britannica, T. Editors of Encyclopaedia (2021). \textit{coronavirus. Encyclopedia Britannica.} https://www.britannica.com/\\science/coronavirus-virus-group

    \hangindent=0.7cm \noindent
    Government of Canada. (2021). https://health-infobase.canada.ca/covid-19/epidemiological-summary-covid-19-cases.html

    \hangindent=0.7cm \noindent
    Mathieu, E., Ritchie, H., Ortiz-Ospina, E., Roser, M., Hasell, J., Appel, C., Giattino, C., \& Rodés-Guirao, L. (2021). A global database of COVID-19 vaccinations. \textit{Nature Human Behaviour}, 5(7), 947-953. https://doi.org/10.1038/\\s41562-021-01122-8

    \hangindent=0.7cm \noindent
    Spiro, T., \& Gee, E. (2020). Thresholds States Must Meet To Control Coronavirus Spread and Safely Reopen. \textit{Center for American Progress.} https://www.americanprogress.org/issues/healthcare/news/2020/05/04/484373/evidence-based-thresholds-states-must-meet-control-coronavirus-spread-safely-reopen-economies/

    \hangindent=0.7cm \noindent
    Statistics Canada. Canada's population clock (real-time model). (2021). https://www150.statcan.gc.ca/n1/pub/71-607-x/71-607-x2018005-eng.htm


\end{document}
